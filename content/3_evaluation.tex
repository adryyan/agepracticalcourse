\section{Auswertung}

Die Schwingungsdauern 
\begin{equation}
    T_i = \frac{1}{2}\left(\frac{t_{i1}}{n_{i1}} + \frac{t_{i2}}{n_{i2}}\right)
\end{equation}
ergeben sich aus den Messgrößen $t_{i1}$ und $t_{i2}$ der beiden Einzelmessungen über alle Perioden, jeweils dividiert durch die Zahl der Schwingungen pro Messung mit der jeweiligen Unsicherheit
\begin{equation}
    \sigma_{T_i} = \sqrt{\frac{\sigma_{t_{i1}}^2}{4 n_{i1}^2} + \frac{\sigma_{t_{i2}}^2}{4 n_{i2}^2}} ,
\end{equation}
welche nach dem Gaußschen Fehlerfortpflanzungsgesetz bestimmt wird.
Durch Einsetzen der gemessenen Werte aus Tabelle \ref{tab:messwerte} ergibt sich
\begin{align*}
    T_0 &= \qty{29,06(3)}{\second} \\
    T_1 &= \qty{31,02(3)}{\second} \\
    T_2 &= \qty{54,60(2)}{\second} .
\end{align*}
Die Federkonstante lässt sich nach Gleichung \eqref{eq:federkonstante} bestimmen
\begin{equation*}
    D = \qty{2,92(9)e-4}{\newton\meter}
\end{equation*}
mit der Unsicherheit\footnote{Dies ist erneut die Gaußsche Fehlerfortpflanzung. Diese muss nicht so ausführlich hingeschrieben werden wie hier.}
\begin{equation*}
    \sigma_D = \sqrt{\left(\frac{\partial D}{\partial m}\cdot \sigma_{m}\right)^2+\left(\frac{\partial D}{\partial R_1}\cdot \sigma_{R_1}\right)^2+\left(\frac{\partial D}{\partial R_2}\cdot \sigma_{R_2}\right)^2+\left(\frac{\partial D}{\partial T_1}\cdot \sigma_{T_1}\right)^2+\left(\frac{\partial D}{\partial T_2}\cdot \sigma_{T_2}\right)^2}
\end{equation*}
und den jeweiligen Ableitungen
\begin{align*}
    \frac{\partial D}{\partial m} &= 4\cdot \pi^2\cdot \frac{R_2^2 - R_1^2}{T_2^2 - T_1^2} \\
    \frac{\partial D}{\partial R_1} &= 4\cdot \pi^2\cdot m\cdot \frac{(-2)\cdot R_1}{T_2^2 - T_1^2} \\
    \frac{\partial D}{\partial R_2} &= 4\cdot \pi^2\cdot m\cdot \frac{2\cdot R_2}{T_2^2 - 
    T_1^2} \\
    \frac{\partial D}{\partial T_1} &= -4\cdot \pi^2\cdot m\cdot \frac{R_2^2 - R_1^2}{(T_2^2-T_1^2)^2}\cdot (-2)\cdot T_1 \\
    \frac{\partial D}{\partial T_2} &= -4\cdot \pi^2\cdot m\cdot \frac{R_2^2 - R_1^2}{(T_2^2 - T_1^2)^2}\cdot 2\cdot T_2 .
\end{align*}
Damit ist es möglich das Torsionsmodul des Stahldrahtes bzw. die Verdrillung nach Gleichung \eqref{eq:torsionsmodul} zu ermitteln:
\begin{equation*}
    G = \qty{7.14(117)e10}{\newton\per\square\meter}
\end{equation*}
Dabei wird die Unsicherheit mit
\begin{equation*}
    \sigma_G = \frac{2}{\pi} \sqrt{\left(\frac{l}{r^4} \cdot \sigma_D\right)^2 + \left(\frac{D}{r^4} \cdot \sigma_l\right)^2 + \left(\frac{4 \cdot D \cdot l}{r^5} \cdot \sigma_r\right)^2}
\end{equation*}
fortgepflanzt.
Mit der Messung ohne zusätzliche Massen $T_0$, der Gleichung \eqref{eq:schwingungsdauer} und der Gaußschen Fehlerfortpflanzung
\begin{equation*}
    \sigma_J = \sqrt{\left(\frac{T_0^2}{4 \pi^2} \cdot \sigma_D\right)^2 + \left(\frac{2 \cdot T_0 \cdot D}{4 \pi^2} \cdot \sigma_{T_0}\right)^2}
\end{equation*}
wird das Trägheitsmoment des Pendels bestimmt:
\begin{equation*}
    J = \qty{6,24(20)e-3}{\kilo\gram\meter\squared}
\end{equation*}
