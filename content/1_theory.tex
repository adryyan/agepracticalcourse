\section{Grundlagen zum Versuch}
Bei diesem Experiment wird ein Drehpendel aufgebaut, bei dem eine Masse symmetrisch an einem Ende eines Metalldrahts aufgehängt ist. Die Pendelbewegung entsteht nach einer anfänglichen Verdrillung des Drahtes um seine Längsachse durch die rücktreibende Kraft, die aus der Torsion eines Drahtes entstammt. 
Solange bei der Verdrillung nicht die Elastizitätsgrenze des Materials überschritten wird, versucht der Draht, wenn keine Kraft mehr auf ihn wirkt, seine ursprüngliche Form wieder anzunehmen.
Die Energie zum Erzeugen des dabei wirkenden Drehmoments ist die gleiche, welche zuvor für die Verdrillung des Drahtes aufgewandt wurde und zwischenzeitlich als elastische Energie im Draht gespeichert war.
Das aus dieser Energie entstehende Drehmoment $M$ ist als
\begin{equation}
    \label{eq:drehmoment1}
    M = -D \cdot \varphi,
\end{equation}
beschrieben, wobei $\varphi$ den Winkel der Verdrillung und $D$ die Federkonstante darstellt \cite{Tipler2015}.
Letztere ergibt sich bei den gewählten Versuchsbedingungen zu
\begin{equation}
    \label{eq:torsionsmodul}
    D = \frac{\pi}{2} \cdot \frac{r^4}{l} \cdot G 
    \quad\Leftrightarrow\quad
    G = \frac{2 \cdot l \cdot D}{\pi \cdot r^4}.
\end{equation}
Hierbei ist $G$ das Torsionsmodul des Drahtes, welches eine materialspezifische Konstante darstellt, $l$ die Länge des Drahts und $r$ dessen Radius \cite{Westphal1971}.

Die Geschwindigkeit, mit der die Ruhelage bei der Pendelbewegung erreicht wird, ist davon abhängig, wie hoch die Trägheit $J$ des Pendels ist.
Je weiter die Massenpunkte des Pendels von der Rotationsachse entfernt sind, desto höher ist das Trägheitsmoment des Pendels und umso geringer ist die Beschleunigung in Richtung der Ruhelage.
Dies lässt sich ebenfalls mit einem Drehmoment beschreiben \cite{Westphal1971}:
\begin{equation}
    \label{eq:drehmoment2}
    M=J\cdot \ddot{\varphi}.
\end{equation}
Aus den beiden Gleichungen \eqref{eq:drehmoment1} und \eqref{eq:drehmoment2} für das Drehmoment ergibt sich die Differentialgleichung der Schwingung zu
\begin{equation}
    \ddot{\varphi} = -\frac{D}{J} \cdot \varphi .
\end{equation}
Die Lösung dieser Gleichung beschreibt die Schwingung des Pendels um die Ruhelage.
Dabei wird periodisch alle elastische Energie in kinetische Energie umgewandelt und umgekehrt.
Diese Pendelbewegung um die Ruhelage findet solange statt, wie nicht alle Energie in Reibungsarbeit umgewandelt worden ist.
Bei der Schwingung beträgt die Schwingungsdauer $T$ \cite{Tipler2015}:
\begin{equation}
    \label{eq:schwingungsdauer}
    T = 2 \cdot \pi \cdot \sqrt{\frac{J}{D}}
    \quad\Leftrightarrow\quad
    J = \frac{T^2 \cdot D}{4 \cdot \pi^2},
    \quad\Leftrightarrow\quad
    D = \frac{4 \cdot \pi^2 \cdot J}{T^2}.
\end{equation}
Auf Basis dieser Gleichung ist es theoretisch möglich, die gesuchte Federkonstante zu berechnen, jedoch ist das Trägheitsmoment $J_0$ des sich drehenden Gestänges am Ende des Drahtes nicht bekannt.
Deswegen werden zwei Messungen mit zusätzlichen Massen, also verschiedenen Trägheitsmomenten $J_1$ und $J_2$, durchgeführt.
Dabei gilt nach dem Steiner'schen Satz $J_i=J_0+m\cdot R^2$ \cite{Walcher2004}.
Werden die Gleichungen für die veränderten Massen ins Verhältnis zueinander gesetzt, kann das unbekannte Drehmoment des Gestänges eliminiert werden:
\begin{align}
    D &= \frac{4 \cdot \pi^2 \cdot J_1}{T_1^2} \quad\quad D = \frac{4 \cdot \pi^2 \cdot J_2}{T_2^2}\nonumber\\
    T_2^2 - T_1^2 &= 4 \pi^2 \cdot \left(\frac{J_0 + m \cdot R_2^2}{D} - \frac{J_0 + m \cdot R_1^2}{D}\right)\nonumber\\
    &= \frac{4 \pi^2 m}{D} \cdot \left(R_2^2 - R_1^2\right)\nonumber\\
    \label{eq:federkonstante}D &= 4 \cdot \pi^2 \cdot m \cdot \frac{R_2^2 - R_1^2}{T_2^2 - T_1^2},
\end{align}
wobei $R$ den Abstand der Masse $m$ zur Rotationsachse darstellt.
Mit einer dritten Messung ohne Zusatzgewichte lässt sich über die Schwingungsdauer das Trägheitsmoment $J_0$ des Pendels nach \eqref{eq:schwingungsdauer} abschließend bestimmen.
